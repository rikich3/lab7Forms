\section{Apreciaciones del laboratorio}

\begin{itemize}
  \item \textbf{Mariel:} Aprendí que Django ofrece una forma poderosa y eficiente de trabajar con bases de datos utilizando modelos, que son esencialmente clases de Python que representan tablas en la base de datos. Esto me permite definir la estructura de los datos de mi aplicación de una manera limpia y legible, y luego utilizar el ORM de Django para realizar operaciones CRUD en esos datos sin tener que escribir consultas SQL directamente. Además, el concepto de aplicaciones en Django me permite organizar mi código de una manera modular y escalable, lo que facilita el mantenimiento y la reutilización del código.

  \item \textbf{Diego:} Me di cuenta de la utilidad del panel de administración de Django (admin) como una forma conveniente de interactuar con los datos de mi aplicación. Puedo registrar mis modelos en el panel de administración y luego gestionar los datos de mi aplicación directamente desde allí sin tener que escribir código adicional. Esto facilita la administración de los datos durante el desarrollo y también puede ser útil en la fase de producción para realizar tareas de administración sin tener que crear interfaces de usuario personalizadas.

  \item \textbf{Jhonatan:} Descubrí que el proceso de migración en Django es una herramienta poderosa para mantener actualizada la estructura de la base de datos de mi aplicación con respecto a los cambios en los modelos. Cuando realizo cambios en mis modelos, como agregar nuevos campos o tablas, puedo generar migraciones usando makemigrations y luego aplicar esas migraciones a mi base de datos usando migrate. Esto me permite mantener la integridad de los datos y garantizar que mi aplicación funcione correctamente incluso después de realizar cambios en los modelos.
  
  \item \textbf{Ricardo:} Aprendí sobre la importancia de la modularidad en Django, especialmente al trabajar con modelos. La capacidad de organizar mi código en aplicaciones me permite mantener una estructura clara y separada para diferentes componentes de mi proyecto. Esto no solo facilita el desarrollo y la colaboración en equipo, sino que también mejora la escalabilidad y la reutilización del código. Además, el uso de el archivo init en la carpeta models para establecer enlaces entre los modelos de diferentes archivos me permite mantener mi código ordenado y fácil de entender.

\end{itemize}