\section{Laboratorio 07}

Este informe detalla el desarrollo y la implementación de un sistema académico utilizando Django, que permite la gestión de cursos, estudiantes y sus notas. El sistema se divide en dos aplicaciones principales: \texttt{courses} y \texttt{users}. En esta seccion, nos centraremos en la aplicación \texttt{courses}, abordando los modelos (\texttt{Course.py, NotasAlumnoPorCurso.py}), vistas (\texttt{views.py}), formularios (\texttt{forms.py}), y plantillas del directorio (\texttt{templates}). A través de ejemplos de código, se explicarán las funcionalidades y la lógica detrás de cada componente.

\section{Modelos de Datos (\texttt{models.py})}

\subsection{Modelo de Curso (\texttt{Course})}

El modelo \texttt{Course} define la estructura de los cursos que se ofrecen en el sistema. Cada curso tiene un código único, un nombre y una descripción.

\begin{lstlisting}[language=Python, caption=Modelo de Curso]
from django.db import models

class Course(models.Model):
    code = models.CharField(max_length=10, unique=True, primary_key=True, default='')
    name = models.CharField(max_length=255, blank=False, null=False)
    description = models.TextField(blank=True, null=False)

    def __str__(self):
        return f"({self.code}) {self.name}"
\end{lstlisting}

\subsubsection{Explicación del Código}

\begin{itemize}
    \item \texttt{code}: Campo de tipo \texttt{CharField} que actúa como clave primaria y debe ser único.
    \item \texttt{name}: Campo de tipo \texttt{CharField} para el nombre del curso.
    \item \texttt{description}: Campo de tipo \texttt{TextField} para la descripción del curso.
    \item \texttt{\_\_str\_\_}: Método que define la representación en cadena del objeto \texttt{Course}.
\end{itemize}

\subsection{Modelo de Notas por Curso (\texttt{NotasAlumnoPorCurso})}

El modelo \texttt{NotasAlumnoPorCurso} registra las notas de los estudiantes en los cursos. Este modelo incluye relaciones con los modelos \texttt{Course} y \texttt{Student}.

\begin{lstlisting}[language=Python, caption=Modelo de Notas por Curso]
from django.db import models
from django.core.validators import MinValueValidator, MaxValueValidator
from courses.models import Course

class NotasAlumnoPorCurso(models.Model):
    id_nota = models.CharField(max_length=100, unique=True, primary_key=True)
    id_curso = models.ForeignKey(Course, related_name='curso', on_delete=models.CASCADE)
    id_alumno = models.ForeignKey('users.Student', related_name='student', on_delete=models.CASCADE)
    nota = models.IntegerField(validators=[MinValueValidator(0), MaxValueValidator(100)])

    def __str__(self):
        return f"({self.id_alumno}) {self.nota}"
\end{lstlisting}

\subsubsection{Explicación del Código}

\begin{itemize}
    \item \texttt{id_nota}: Campo de tipo \texttt{CharField} que actúa como clave primaria y debe ser único.
    \item \texttt{id_curso}: Campo de tipo \texttt{ForeignKey} que relaciona una nota con un curso.
    \item \texttt{id_alumno}: Campo de tipo \texttt{ForeignKey} que relaciona una nota con un estudiante.
    \item \texttt{nota}: Campo de tipo \texttt{IntegerField} que almacena la nota del estudiante, validada entre 0 y 100.
\end{itemize}

\section{Vistas (\texttt{views.py})}

Las vistas manejan la lógica de las solicitudes y respuestas del sistema. En esta sección, se describen las vistas para la creación y listado de cursos, así como la adición de notas.

\subsection{Vista de Inicio (\texttt{index})}

\begin{lstlisting}[language=Python, caption=Vista de Inicio]
from django.shortcuts import render

def index(request):
    return render(request, 'sistema/index.html')
\end{lstlisting}

\subsection{Vista para Crear Curso (\texttt{create\_curso})}

\begin{lstlisting}[language=Python, caption=Vista para Crear Curso]
from django.shortcuts import render, redirect
from .forms import CourseForm

def create_curso(request):
    if request.method == 'POST':
        form = CourseForm(request.POST)
        if form.is_valid():
            form.save()
            return redirect('list_cursos')
    else:
        form = CourseForm()
    return render(request, 'sistema/create_curso.html', {'form': form})
\end{lstlisting}

\subsubsection{Explicación del Código}

\begin{itemize}
    \item Si el método de la solicitud es \texttt{POST}, se procesa el formulario.
    \item Si el formulario es válido, se guarda el nuevo curso y se redirige a la lista de cursos.
    \item Si el método es \texttt{GET}, se muestra un formulario vacío.
\end{itemize}

\subsection{Vista para Listar Cursos (\texttt{list\_cursos})}

\begin{lstlisting}[language=Python, caption=Vista para Listar Cursos]
from django.shortcuts import render
from .models import Course

def list_cursos(request):
    cursos = Course.objects.all()
    return render(request, 'sistema/list_cursos.html', {'cursos': cursos})
\end{lstlisting}

\subsubsection{Explicación del Código}

\begin{itemize}
    \item Recupera todos los cursos de la base de datos y los pasa al template para ser mostrados.
\end{itemize}

\subsection{Vista para Agregar Nota (\texttt{addGrade})}

\begin{lstlisting}[language=Python, caption=Vista para Agregar Nota]
from django.shortcuts import render, redirect
from .forms import NotasAlumnoPorCursoForm

def addGrade(request):
    if request.method == 'POST':
        form = NotasAlumnoPorCursoForm(request.POST)
        if form.is_valid():
            form.save()
            return redirect('success')
    else:
        form = NotasAlumnoPorCursoForm()
    return render(request, 'sistema/add_grade.html', {'form': form})
\end{lstlisting}

\subsubsection{Explicación del Código}

\begin{itemize}
    \item Similar a \texttt{create\_curso}, pero para la adición de notas.
\end{itemize}

\section{Formularios (\texttt{forms.py})}

Los formularios son esenciales para la entrada y validación de datos en Django.

\subsection{Formulario de Curso (\texttt{CourseForm})}

\begin{lstlisting}[language=Python, caption=Formulario de Curso]
from django import forms
from .models import Course

class CourseForm(forms.ModelForm):
    class Meta:
        model = Course
        fields = ['name', 'code', 'description']
\end{lstlisting}

\subsubsection{Explicación del Código}

\begin{itemize}
    \item Se especifica el modelo y los campos a incluir en el formulario.
\end{itemize}

\subsection{Formulario de Notas por Curso (\texttt{NotasAlumnoPorCursoForm})}

\begin{lstlisting}[language=Python, caption=Formulario de Notas por Curso]
from django import forms
from .models import NotasAlumnoPorCurso

class NotasAlumnoPorCursoForm(forms.ModelForm):
    class Meta:
        model = NotasAlumnoPorCurso
        fields = ['id_curso', 'id_alumno', 'nota']
        labels = {
            'id_alumno': 'Alumno',
            'id_curso': 'Curso',
            'nota': 'Nota'
        }

    def clean_grade(self):
        grade = self.cleaned_data.get('grade')
        if grade < 0 or grade > 100:
            raise forms.ValidationError('La nota debe estar entre 0 y 100')
        return grade
\end{lstlisting}

\subsubsection{Explicación del Código}

\begin{itemize}
    \item Se especifican el modelo y los campos a incluir.
    \item Se incluye un método de validación para asegurarse de que la nota esté dentro del rango permitido.
\end{itemize}

\section{Plantillas (\texttt{templates})}

Las plantillas definen cómo se presenta la información en la interfaz de usuario.

\subsection{Plantilla para Crear Curso (\texttt{create\_curso.html})}

\begin{lstlisting}[language=HTML, caption=Plantilla para Crear Curso]
<!DOCTYPE html>
<html>
<head>
    <meta charset="UTF-8" />
    <meta name="viewport" content="width=device-width, initial-scale=1.0" />
    <title>Crear Curso</title>
</head>
<body>
    <header>Sistema Académico</header>
    <main>
        <a href="">Regresar al menú</a>
        <h2>Crear Curso</h2>
        <form method="post">
            
            <table>
                {{ form.as_table }}
            </table>
            <button type="submit">Crear</button>
        </form>
        <a href="">Lista de cursos</a>
    </main>
</body>
</html>
\end{lstlisting}

\subsection{Plantilla para Listar Cursos (\texttt{list\_cursos.html})}

\begin{lstlisting}[language=HTML, caption=Plantilla para Listar Cursos]
<!DOCTYPE html>
<html>
<head>
    <meta charset="UTF-8" />
    <meta name="viewport" content="width=device-width, initial-scale=1.0" />
    <title>Lista de Cursos</title>
</head>
<body>
    <header>Sistema Académico</header>
    <main>
        <a href="">Regresar al menú</a>
        <h2>Lista de cursos</h2>
        <table>
            <tr>
                <th>Código</th>
                <th>Nombre</th>
                <th>Descripción</th>
            </tr>
            
            <tr>
                <td>{{ curso.code }}</td>
                <td>{{ curso.name }}</td>
                <td>{{ curso.description }}</td>
            </tr>
            
        </table>
        <a href="">Crear un curso</a>
    </main>
</body>
</html>
\end{lstlisting}

\subsection{Plantilla de Índice (\texttt{index.html})}

\begin{lstlisting}[language=HTML, caption=Plantilla de Índice]
<!DOCTYPE html>
<html lang="en">
<head>
    <meta charset="UTF-8" />
    <meta name="viewport" content="width=device-width, initial-scale=1.0" />
    
    <link rel="stylesheet" href="" />
    <title>Índice</title>
</head>
<body>
    <header>
        <h1>Sistema Académico</h1>
    </header>
    <main>
        <section class="opciones">
            <h2>Opciones Disponibles</h2>
            <div class="opcion">
                <a href="">Lista de Alumnos</a>
            </div>
            <div class="opcion">
                <a href="">Crear Alumno</a>
            </div>
            <div class="opcion">
                <a href="">Lista de Cursos</a>
            </div>
            <div class="opcion">
                <a href="">Crear Curso</a>
            </div>
        </section>
    </main>
</body>
</html>
\end{lstlisting}