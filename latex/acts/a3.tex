\section{Users app}

Ahora nos centraremos en la aplicación \texttt{users}, abordando los modelos (\texttt{models.py}), vistas (\texttt{views.py}), formularios (\texttt{forms.py}), y plantillas (\texttt{templates}). A través de ejemplos de código, se explicarán las funcionalidades y la lógica detrás de cada componente.

\section{Modelos de Datos (\texttt{models.py})}

\subsection{Modelo de Usuario (\texttt{User})}

El modelo \texttt{User} es una clase abstracta que define los campos comunes a todos los usuarios del sistema, como el ID, nombre y correo electrónico. Esta clase es la base para otros modelos específicos de usuario, como \texttt{Student}.

\begin{lstlisting}[language=Python, caption=Modelo \texttt{User}]
# users/models/User.py
from django.db import models
import uuid

class User(models.Model):
    id = models.UUIDField(primary_key=True, default=uuid.uuid4, editable=False)
    name = models.CharField(max_length=255, null=False, blank=False)
    email = models.EmailField(unique=True, null=False, blank=False)

    class Meta:
        abstract = True  # Esto hace que User sea una clase base abstracta

    def __str__(self):
        return self.name
\end{lstlisting}

\subsubsection{Explicacion del Codigo: }

\begin{itemize}
\item \texttt{id}: Campo de tipo \texttt{UUIDField} que actúa como clave primaria y se genera automáticamente.
\item \texttt{name}: Campo de tipo \texttt{CharField} para el nombre del usuario.
\item \texttt{email}: Campo de tipo \texttt{EmailField} que debe ser único.
\item \texttt{abstract = True}: Indica que \texttt{User} es una clase abstracta y no se creará una tabla para esta clase en la base de datos.

\end{itemize}

\subsection{Modelo de Estudiante (\texttt{Student})}

El modelo \texttt{Student} hereda de \texttt{User} y añade una relación con el modelo \texttt{Course}, permitiendo asignar cursos a los estudiantes.

\begin{lstlisting}[language=Python, caption=Modelo \texttt{Student}]
# users/models/student.py
from django.db import models
from .User import User
from courses.models import Course

class Student(User):
    courses = models.ForeignKey(Course, related_name='students', on_delete=models.CASCADE, null=True, blank=True)
\end{lstlisting}

\subsubsection{Explicacion del Codigo: }

- \texttt{courses}: Campo de tipo \texttt{ForeignKey} que relaciona al estudiante con un curso. Permite la relación de muchos a uno (muchos estudiantes pueden estar inscritos en un curso).

\section{Vistas (\texttt{views.py})}

Las vistas manejan la lógica de las solicitudes y respuestas del sistema. En esta sección, se describe la vista para la creación de estudiantes.

\subsection{Vista para Agregar Estudiante (\texttt{addStudent})}

\begin{lstlisting}[language=Python, caption=Vista para Agregar Estudiante]
# users/views.py
from django.shortcuts import render, redirect
from .forms import StudentForm

def addStudent(request):
    if request.method == 'POST':
        form = StudentForm(request.POST)
        if form.is_valid():
            form.submit()
            return render(request, 'success.html', {})
    else:
        form = StudentForm()
    return render(request, 'form.html', {'form': form})
\end{lstlisting}

\subsubsection{Explicacion del Codigo: }

\begin{itemize}
\item Si el método de la solicitud es \texttt{POST}, se procesa el formulario.
\item Si el formulario es válido, se guarda el nuevo estudiante y se muestra una página de éxito.
\item Si el método es \texttt{GET}, se muestra un formulario vacío.
\end{itemize}

\section{Formularios (\texttt{forms.py})}

Los formularios son esenciales para la entrada y validación de datos en Django.

\subsection{Formulario de Estudiante (\texttt{StudentForm})}

\begin{lstlisting}[language=Python, caption=Formulario de Estudiante]
# users/forms.py
from django import forms
from .models import Student

class StudentForm(forms.ModelForm):
    class Meta:
        model = Student
        fields = ['name', 'email']

    def clean(self):
        cleaned_data = super().clean()
        student_id = cleaned_data.get('id')
        if Student.objects.filter(id=student_id).exists():
            raise forms.ValidationError('Ya existe un estudiante con ese id')
        return cleaned_data

    def submit(self, commit=True):
        instance = super().save(commit=False)
        if commit:
            instance.save()
        return instance
\end{lstlisting}

\subsubsection{Explicacion del Codigo: }

\begin{itemize}
\item Se especifica el modelo y los campos a incluir en el formulario.
\item \texttt{clean}: Método de validación que verifica si ya existe un estudiante con el mismo ID.
\item \texttt{submit}: Método para guardar la instancia del formulario.
\end{itemize}

\section{Plantillas (\texttt{templates})}

Las plantillas definen cómo se presenta la información en la interfaz de usuario.

\subsection{Plantilla para el Formulario (\texttt{form.html})}

\begin{lstlisting}[language=HTML, caption=Plantilla para el Formulario]
<!DOCTYPE html>
<html lang="en">
<head>
    <meta charset="UTF-8">
    <meta name="viewport" content="width=device-width, initial-scale=1.0">
    <title>Registrando Usuario</title>
</head>
<body>
    <h1>{{ Label|upper }}</h1>
    <form method="post">
        
        {{ form.as_p }}
        <button type="submit">Registrar</button>
    </form>
</body>
</html>
\end{lstlisting}

\subsection{Plantilla de Éxito (\texttt{success.html})}

\begin{lstlisting}[language=HTML, caption=Plantilla de Éxito]
<!DOCTYPE html>
<html lang="en">
<head>
    <meta charset="UTF-8">
    <meta name="viewport" content="width=device-width, initial-scale=1.0">
    <title>Éxito</title>
</head>
<body>
    <h1>REALIZADO CON ÉXITO</h1>
</body>
</html>
\end{lstlisting}

\section{Configuración de URLs (\texttt{urls.py})}

La configuración de URLs es esencial para enrutar las solicitudes a las vistas correspondientes.

\begin{lstlisting}[language=Python, caption=Configuración de URLs]
# urls.py
from django.contrib import admin
from django.urls import path
from users.views import addStudent
from courses.views import create_curso, list_cursos, addGrade, index

urlpatterns = [
    path('admin/', admin.site.urls),
    path('sistema/addStudent/', addStudent, name='addStudent'),
    path('sistema/', index, name='index'),
    path('/', index, name='index'),
    path('sistema/create_curso/', create_curso, name='create_curso'),
    path('sistema/list_cursos/', list_cursos, name='list_cursos'),
]
\end{lstlisting}

\subsubsection*{Explicacion del Codigo:}

\begin{itemize}
\item \texttt{path('admin/', admin.site.urls)}: URL para la administración de Django.
\item \texttt{path('sistema/addStudent/', addStudent, name='addStudent')}: URL para agregar un nuevo estudiante.
\item \texttt{path('sistema/', index, name='index')}: URL para la página de inicio.
\item \texttt{path('sistema/create\_curso/', create\_curso, name='create\_curso')}: URL para crear un curso.
\item \texttt{path('sistema/list\_cursos/', list\_cursos, name='list\_cursos')}: URL para listar los cursos.
\end{itemize}